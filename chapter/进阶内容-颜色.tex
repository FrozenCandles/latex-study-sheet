\subsection{颜色}

\begin{enumerate}

\item 简单的文本颜色:\boxforpkg{xcolor} 宏包提供了创建颜色的工具。\boxforcmd{\\color{}} 是一个声明,它能使块中接下来所有文本(包括公式)变成参数的颜色,例如声明 \verb|\color{blue}| {\color{blue}会使接下来所有文本变为蓝色。}

\item 页面颜色:使用 \boxforcmd{\\pagecolor{}} 会更改当前及其之后页面的颜色,在某一页使用 \boxforcmd{\\nopagecolor} 可以暂时去除该页面施加的颜色效果。

\fcolorbox{black}{black}{%
\begin{minipage}{\dimexpr0.33\textwidth-2\fboxrule-2\fboxsep\relax}
\color{white}
这是一个使用 \lstinline[language=TeX]{\\pagecolor{black}} 和 \lstinline[language=TeX]{\\color{white}} 的页面局部效果。
\end{minipage}}

\item 颜色盒子:\boxforcmd{\\fcolorbox{}{}{}} 可以创建一个颜色盒子,为文本添加带有颜色的边框和背景,三个参数分别是边框颜色、背景色和文本内容。

\fcolorbox{blue}{yellow}{这是一段具有 blue 边框、yellow 背景的文本。}

\item 局部颜色:使用 \boxforcmd{\\textcolor{}{}} 可以创建\textcolor{red}{局部的颜色},两个参数分别为颜色名和影响的文本。它常用于公式环境中,例如:

\begin{tcolorbox}[sidebyside]
\begin{lstlisting}
\[ \lim_{\textcolor{red}{\Delta}\to 0}
(1+\textcolor{red}{\Delta})^\frac
{1}{\textcolor{red}{\Delta}} =e\]
\end{lstlisting} 

\tcblower

\[
\lim_{\textcolor{red}{\Delta} \to 0}
(1+\textcolor{red}{\Delta})
^\frac{1}{\textcolor{red}{\Delta}} = e
\]
\end{tcolorbox}

\item 颜色名:可以直接使用颜色名指代颜色,\boxforpkg{xcolor} 宏包可以直接使用的颜色名约有20个。使用宏包的 \boxforcmd{[dvipsnames]} 参数可以额外使用约70个颜色名,例如\;{\color{BrickRed}\verb|BrickRed|} 和\;{\color{Periwinkle}\verb|Periwinkle|} 这样的颜色名。

进一步使用宏包的 \boxforcmd{[x11names]} 参数可以加载300多种颜色名,例如\;{\color{Aquamarine3}\verb|Aquamarine3|} 和\;{\color{Goldenrod2}\verb|Goldenrod2|} 这样的颜色名。

\item 自定义颜色:在导言区使用 \boxforcmd{\\definecolor{}{}{}} 可以自定义颜色,三个参数分别是自定义的颜色名、颜色格式、颜色值。颜色值的形式取决于颜色格式,可用的颜色格式有:

\begin{itemize}[itemsep=0pt]
    \item \verb|RGB| :红绿蓝混色,每种色值取值范围为 0--255 ,用逗号分隔
    \item \verb|rgb| :同上,但每种色值取值范围为 0--1.0
    \item \verb|cmyk| :彩印标准颜色,每种色值取值范围为 0--1.0
    \item \verb|gray| :灰度颜色,取值范围为 0--1.0
    \item \verb|HTML| :六位十六进制 RGB 颜色值
    \item \verb|wave| :对应波长的颜色,取值为 363--814 
\end{itemize}

对于如下定义:

\begin{tcolorbox}[colback=white]
\begin{lstlisting}[language=TeX]
\definecolor{FrozenBlue}{RGB}{157,234,242}
\definecolor{CandleRed}{HTML}{F57267}
\end{lstlisting}
\end{tcolorbox}

这是\;{\color{FrozenBlue}\verb|FrozenBlue|} 和\;{\color{CandleRed}\verb|CandleRed|} 的呈现效果。

如果 \verb|\definecolor| 定义的颜色名和现有颜色名重复,会覆盖之前定义的颜色名。\boxforcmd{\\providecolor} 命令用法类似,但发现重复定义后会放弃新的定义。

\item 自定义混合颜色:\boxforcmd{\\colorlet{}{}} 提供了通过混合颜色来定义颜色的一种形式,被混合颜色和混合比例均以感叹号 \boxforcmd{!} 结尾,例如:

\begin{lstlisting}[language=TeX]
\colorlet{PurplePink}{red!60!blue!40!}
\end{lstlisting}

会使用 60\% 红色和 40\% 蓝色混合出\;{\color{PurplePink}\verb|PurplePink|} 颜色。

\end{enumerate}