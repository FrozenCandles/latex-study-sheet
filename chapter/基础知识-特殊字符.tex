\subsection{特殊符号}
\begin{enumerate}

\item 空格:在 \LaTeX 中,字符间的空白会自动调整。

不带参数的命令后面的空格,要用空的花括号对加上空格来表示,否则空格会被忽略,例如:

\begin{tcolorbox}[sidebyside]
\begin{lstlisting}
\LaTeX without \{\}
\LaTeX{} with \{\}
\end{lstlisting} 

\tcblower

\LaTeX without \{\}

\LaTeX{} with \{\}
\end{tcolorbox}

\item 引号:英文单引号分左右引号。左单引号用重音符 \boxforcmd{`} 表示,右单引号用普通引号 \boxforcmd{'} 表示,左双引号用连续两个重音符 \boxforcmd{``} 表示,右双引号用连续两个单引号 \boxforcmd{''} 表示。

英文下的引号嵌套需要借助 \boxforcmd{\\thinspace} 命令分隔,例如:

\begin{tcolorbox}[sidebyside]
\begin{lstlisting}
``\thinspace`single' quotes''
\end{lstlisting}

\tcblower

``\thinspace`single' quotes''
\end{tcolorbox}

中文引号可以直接输入。

\item 短横:英文短横有 3 种:

\begin{itemize}
    \item 连字符:用一个短横 \boxforcmd{-} 表示,如 clear-cut ; 
    \item 数字起止符:用两个短横 \boxforcmd{--} 表示,如 page 1--2 ;
    \item 破折号:用三个短横 \boxforcmd{---} 表示,如 Look---It's a dash 。
\end{itemize}

中文破折号可以直接输入。

\item 省略号:英文省略号用 \boxforcmd{\\ldots} 符号表示,效果为 \ldots 。中文省略号可以直接输入。

\item 保留字符:以下字符在 \LaTeX 中具有特殊含义,不能直接作为文档中的一个字符:

\begin{tcolorbox}
\lstinputlisting[language=C]{./resource/lstcode/01-special-chars.tex}
\end{tcolorbox}

除反斜杠外,其余字符均能用反斜杠的形式转义输出:

\begin{tcolorbox}[sidebyside]
\begin{lstlisting}[language=C]
\# \$ \% \^{} \& \_ \{ \}
\end{lstlisting} 

\tcblower

\; \# \$ \% \^{} \& \_ \{ \}
\end{tcolorbox}

反斜杠 \boxforcmd{\\} 可以使用以下方式得到:

\begin{tcolorbox}
\begin{lstlisting}
\textbackslash $\backslash$
\end{lstlisting}
\end{tcolorbox}


\end{enumerate}