\subsection{段落处理}
\begin{enumerate}

\item 文本对齐:有左中右三种对齐方式,如下表:

\begin{tcolorbox}[colback=white]
\centering
\begin{tabular}{cccc}
    对齐方式 & 声明形式 & 环境形式 & 参数形式 \\
\hline
    居中对齐 & \verb|\centering| & \verb|center| 环境 & \verb|\centerline{}| \\
    左对齐 & \verb|\raggedright| & \verb|flushleft| 环境 & \verb|\leftline{}| \\
    右对齐 & \verb|\raggedleft| & \verb|flushright| 环境 & \verb|\rightline{}| \\
\end{tabular}
\end{tcolorbox}

声明形式的对齐会影响接下来所有文本的对齐方式,环境形式只影响环境内的对齐方式,参数形式只影响参数内文本的对齐方式。

\item 换行:\LaTeX 中的多个空格会被视为一个,多个换行符也会被视为一个。使用两个回车可用于在正文中换行。使用两个反斜杠 \boxforcmd{\\\\} 用于强制换行。

\item 段落与缩进:使用 \boxforcmd{\\par} 可以生成一个带缩进的新段。

\item 换页:使用 \boxforcmd{\\newpage} 开始新的一页。 



\end{enumerate}