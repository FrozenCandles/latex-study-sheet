\subsection{表格}
\begin{enumerate}

\item 基本表格结构:表格一般都直接用 \LaTeX 命令绘制。制作表格需要确定表格的行、列对齐模式和表格线,由 \boxforenv{tabular} 环境完成:

\begin{tcolorbox}[sidebyside]
\begin{lstlisting}
\begin{tabular}{|rrr|}
    \hline
        angle & $\sin$ & $\cos$ \\
    \hline
            0 &   0  & 1 \\
    $\pi/2$ &   1  & 0 \\
    \hline
\end{tabular}
\end{lstlisting} 

\tcblower
\centering
\begin{tabular}{|rrr|}
    \hline
          angle & $\sin$ & $\cos$ \\
    \hline
              0 &   0  & 1 \\
      $\pi / 2$ &   1  & 0 \\
    \hline
\end{tabular}
\end{tcolorbox}

\boxforenv{tabular} 环境有一个参数,声明了表格中列的模式:\boxforcmd{\|rrr\|} 表示表格有三列,都是右对齐(right),在第一列前面和第三列后面各有一条垂直的表格线。在 \verb|tabular| 环境内部,行与行之间用换行命令 \boxforcmd{\\\\} 隔开,每行内部的表项则用符号 \boxforcmd{&} 隔开。表格中的横线则是用命令 \boxforcmd{\\hline} 产生的。

\item 表格的对齐:基本的对齐参数 \boxforcmd{lcr} 表示左中右对齐,表格宽度会自动调整,但不会自动换行,可能超出页面宽度。可以用 \boxforcmd{p{}} 形式指定某一列的宽度,参数内为宽度值,这时自动左对齐。

可以用可选参数表示文本垂直对齐方式,\boxforcmd{tcb} 分别表示顶端对齐、垂直居中、底端对齐,默认顶端对齐。

\item 表格线:在必选参数内连续使用两个竖线 \boxforcmd{||} 表示绘制一条竖直双线,使用 \boxforcmd{@{}} 在每行对应绘制的竖直表格线替换为参数里的符号,参数为空表示不画竖直表格线。

命令 \boxforcmd{\\hline} 用于绘制一根水平表线,而命令 \boxforcmd{\\cline{c1-c2}} 仅绘制从 \verb|c1| 列到 \verb|c2| 列的水平表线。这两个命令后面不用加换行符,因为它算当前行的顶部。两个连续的 \verb|\hline| 可以画水平双线,但是这种方式制造的双线与竖直表线的相交效果不好。

\item 竖向表线与拆分单元格:在单元格内使用 \boxforcmd{\\vline} 命令仅画出等于所在行行高的竖直线。该命令常常用于横向拆分单元格。

\item 表格的高级对齐:加载 \boxforpkg{array} 宏包可以使表格使用更多对齐效果,例如 \boxforcmd{m{}} 在指定某列宽度时,也会使该行的其余部分垂直居中;\boxforcmd{b{}} 在指定某列宽度时,也会使该行的其余部分底端对齐。例如:

\begin{tcolorbox}[colback=white]
    \centering
    \begin{tabular}{|c|p{5.6em}|}
        \hline
            align & this column uses \verb|p{}|\\
        \hline
    \end{tabular}
    \begin{tabular}{|c|m{5.6em}|}
        \hline
            align & this column uses \verb|m{}|\\
        \hline
    \end{tabular}
    \begin{tabular}{|c|b{5.6em}|}
        \hline
            align & this column uses \verb|b{}|\\
        \hline
    \end{tabular}
\end{tcolorbox}

\item 列格式:\boxforpkg{array} 宏包还提供了 \boxforcmd{>{}} 和 \boxforcmd{<{}} ,分别用在 \verb|lcrpmb| 参数前后,可以使该列每个单元格开头、结尾都套上参数内的命令或声明。例如以下表格参数:

\begin{tcolorbox}[sidebyside]
\begin{lstlisting}
{|>{\bfseries}c|
  >{\raggedleft\arraybackslash}p{6cm}|}
\end{lstlisting} 

\tcblower

\begin{center}
    \begin{tabular}{|>{\bfseries}c|>{\raggedleft\arraybackslash}p{6cm}|}
        \hline
        ohhh & It's a very very very long loong looong loooong looooong line\\
        \hline
    \end{tabular}
\end{center}
\end{tcolorbox}

使第一列单元格使用加粗字系,第二列单元格左边不平齐(就是右对齐)。其中 \boxforcmd{\\arraybackslash} 是为避免行末的 \verb|\\| 出现异常。

\item 自定义列格式:可以在导言区通过自定义列格式命令 \boxforcmd{\\newcolumntype{}{}} 将复杂的 \verb|lcrpmb| 与 \verb|>{}| 和 \verb|<{}| 组合成一个格式,第一个参数是新的格式名(必须是单字母),第二个参数是组合格式。例如,在导言区定义:

\begin{tcolorbox}
\begin{lstlisting}
\newcolumntype{f}{>{\(}c<{\)}}
\end{lstlisting}
\end{tcolorbox}

则表格环境的参数中为 \boxforcmd{f} 的列处于行内数学环境 \verb|\( ... \)| 并居中。

\item 小数对齐的列格式:宏包 \boxforpkg{siunitx} 提供了小数对齐的列格式,在导言区使用以下命令可以保留两位小数点:

\begin{tcolorbox}
\begin{lstlisting}
\sisetup{
  round-mode=places,
  round-precision=2,
}
\end{lstlisting}
\end{tcolorbox}

然后便可以使用 \boxforcmd{S} 格式表示该列需要对齐小数点,效果为:

\begin{tcolorbox}[sidebyside]
\begin{lstlisting}
\begin{tabular}{|c|S|} 
    \hline
    A & 1.234 \\
    B & 12.34 \\
    C & 123.4 \\
    D & 12 \\ \hline 
\end{tabular}
\end{lstlisting} 

\tcblower

\begin{tabular}{|c|S|} 
    \hline
    A & 1.234 \\
    B & 12.34 \\
    C & 123.4 \\
    D & 12 \\
    \hline 
\end{tabular}
\end{tcolorbox}

\item 表格与浮动:\boxforenv{tabular} 环境得到的也是一个比较大的盒子。一般也放在浮动环境 \boxforenv{table} 中,参数与大体的使用格式也与 \verb|figure| 环境类似:

\begin{tcolorbox}
\begin{lstlisting}
\begin{table}[H]
  % tabular
\end{table}
\end{lstlisting}
\end{tcolorbox}

可选参数 \boxforcmd{[H]} 表示不浮动(Here)。该选项是由 \boxforpkg{float} 宏包提供的特殊功能。

\item 跨行列的单元格:需要 \boxforpkg{multirow} 宏包支持,分别使用 \boxforcmd{\\multirow{}{}{}} 和 \boxforcmd{\\multicolumn{}{}{}} 命令跨行和跨列。三个参数中:

\begin{itemize}
    \item 第一个参数表示跨越(总共包含)的行列数;
    \item 第二个参数同 \boxforenv{tabular} 环境的参数(对齐和画线),也可以使用 \boxforcmd{*} 表示自动适应宽度;
    \item 最后一个参数是单元格内填入的文本。
\end{itemize}

跨行列后,被跨位置的单元格仍应正常表示且内容留空,它们会被覆盖以实现跨行列。

\begin{tcolorbox}[sidebyside]
\begin{lstlisting}
\begin{tabular}{cc}
  \hline
  \multirow{2}{*}{multirow} & row1-2 \\
                            & row2-2 \\
  \hline
\end{tabular}
\end{lstlisting} 

\tcblower

\begin{tabular}{cc}
    \hline
    \multirow{2}{*}{multirow} & row1-2\\
        & row2-2 \\
    \hline
\end{tabular}
\end{tcolorbox}

同时跨行跨列注意嵌套关系 \boxforcmd{\\multirow} 命令放在 \boxforcmd{\\multicolumn} 内部。

一个非常复杂的综合示例:

\begin{tcolorbox}[sidebyside]
\begin{lstlisting}
\begin{tabular}{|c|c|c|c|c|}
  \hline
  \multirow{2}{*}{multi-row} &
  \multicolumn{2}{c|}{multi-col} &
  \multicolumn{2}{c|}{
    \multirow{2}{*}{multi-row\&col}}\\
  \cline{2-3}
    & col-1 & col-2 & 
    \multicolumn{2}{c|}{} \\
  \hline
  item1&item2&item3&item4&item5\\
  \hline
\end{tabular}
\end{lstlisting}

\tcblower

\begin{center}
\begin{tabular}{|c|c|c|c|c|}
    \hline
    \multirow{2}{*}{multi-row} &
    \multicolumn{2}{c|}{multi-col} &
    \multicolumn{2}{c|}{\multirow{2}{*}{multi-row\&col}} \\
    \cline{2-3}
      & col-1 & col-2 & \multicolumn{2}{c|}{} \\
    \hline
    item1 & item2 & item3 & item4 & item5 \\
    \hline
\end{tabular}
\end{center}
\end{tcolorbox}

\boxforcmd{\\cline{2-3}} 用于在 multi-col 下画一条第 2 栏位到第 3 栏位的边框线。空的 \boxforcmd{\\multicolumn} 用于修改同时跨行列单元格的边框线问题。

\item 拆分单元格:拆分单元格的需求可以通过表格嵌套实现。

\item 单元格换行:宏包 \boxforpkg{makecell} 提供了在单元格内换行方式,使用 \boxforcmd{\\makecell{}} 命令可以在参数内使用 \verb|\\| 方便地换行:

\begin{tcolorbox}[sidebyside]
\begin{lstlisting}
\begin{tabular}{|c|c|c|} 
    \hline col1 & 
    \makecell{line1 \\ line2} & 
    col3 \\ \hline
\end{tabular}
\end{lstlisting}

\tcblower

\begin{tabular}{|c|c|c|} 
  \hline col1 & 
  \makecell{row1\\row2}& 
  col3 \\ \hline
\end{tabular}
\end{tcolorbox}

还可以配合可选参数使用 \boxforcmd{[lcrtb]} 之一指定单元格对齐方式。

\item 水平表线宽:\boxforpkg{makecell} 宏包还提供了 \boxforcmd{\\Xhline{}} 和 \boxforcmd{\\Xcline{}{}} 命令,可以通过最后一个额外的参数指定水平表线宽。以下是一个模仿三线表的示例:

\begin{tcolorbox}[sidebyside]
\begin{lstlisting}
\begin{tabular}{ccc}
  \Xhline{2pt}
  \multirow{2}*{X} &
  \multicolumn{2}{c}{Y} \\
  \Xcline{2-3}{0.4pt} & Left & Right \\
  \Xhline{1pt}
  a & A & C \\ b & B & D \\
  \Xhline{2pt}
\end{tabular}
\end{lstlisting}

\tcblower

\begin{tabular}{ccc}
    \Xhline{2pt}
    \multirow{2}*{X} &
    \multicolumn{2}{c}{Y}\\
    \Xcline{2-3}{0.4pt}
    & Left & Right \\
    \Xhline{1pt}
    a & A & C \\
    b & B & D \\
    \Xhline{2pt}
\end{tabular}\
\end{tcolorbox}

\item 分割表头:\boxforpkg{diagbox} 宏包提供了分割表头的命令 \boxforcmd{\\diagbox} 。该命令支持两个或三个参数,分别表示将表头分割成两部分或三部分,例如:

\begin{tcolorbox}[sidebyside]
\begin{lstlisting}
\begin{tabular}{c|c}
  \hline \diagbox{X}{data}{Y} & $y$ \\ 
  \hline $x$ & 10 \\ \hline 
\end{tabular}
\end{lstlisting}

\tcblower

\begin{tabular}{|c|c|}
    \hline \diagbox{X}{data}{Y} & $y$ \\ 
    \hline $x$ & 10 \\ \hline 
\end{tabular}
\end{tcolorbox}

\end{enumerate}
