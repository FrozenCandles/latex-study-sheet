\section{复合公式}

\subsection{矩阵}

\begin{enumerate}
    
\item 基本矩阵:使用 \verb|array| 环境可以得到表格一样的横竖对齐,可以用于对齐公式或排版矩阵。这些具有对齐的环境一般都使用 \boxforcmd{\\\\} 切换到下一行,同一行内用 \boxforcmd{&} 切换到下一列:

\begin{tcolorbox}[sidebyside]
\begin{lstlisting}
\[ \begin{array}{ccc}
    x_{11} & ax + b & x_{13} \\
    x_{21} & e^x & x_{23} 
\end{array} \]
\end{lstlisting} 

\tcblower

\[ \begin{array}{ccc}
    x_{11} & ax + b & x_{13} \\
    x_{21} & e^x & x_{23} 
\end{array} \]
\end{tcolorbox}

额外的参数指明了每一列的水平对齐方式都是居中对齐(center),还可以使用 \boxforcmd{l} 指定左对齐以及使用 \boxforcmd{r} 指定右对齐。

\item 通用矩阵:宏包 \boxforpkg{amsmath} 提供了通用的 \boxforenv{matrix} 矩阵环境,无需手动指明对齐方式,其余用法一致:

\begin{tcolorbox}[sidebyside]
\begin{lstlisting}
\[\begin{matrix} 
  1 & 0 \\ 0 & 1
\end{matrix}\]
\end{lstlisting} 

\tcblower

\[\begin{matrix} 1 & 0 \\ 0 & 1
\end{matrix}\]
\end{tcolorbox}

\item 矩阵的边界符:可以用左右高度自适应的定界符为矩阵加上边界符。宏包 \boxforpkg{amsmath} 以 \boxforcmd{pbBvV} 开头的各种 \boxforenv{matrix} 环境可以为矩阵两侧加上各种边界符,以下是按顺序展示的各个边界符效果:

\[ \begin{pmatrix} a&b\\c&d \end{pmatrix} \quad
   \begin{bmatrix} a&b\\c&d \end{bmatrix} \quad
   \begin{Bmatrix} a&b\\c&d \end{Bmatrix} \quad
   \begin{vmatrix} a&b\\c&d \end{vmatrix} \quad
   \begin{Vmatrix} a&b\\c&d \end{Vmatrix} \]

\item 行内小矩阵:宏包 \boxforpkg{amsmath} 提供了 \boxforenv{smallmatrix} 环境,可以得到行内公式的小矩阵,加上自适应的左右括号后为 $\left(\begin{smallmatrix} a&b\\c&d \end{smallmatrix}\right)$ 效果。

\item 矩阵的边界符与对齐:宏包 \boxforpkg{mathtools} 提供了各种带星的 \boxforenv{matrix*} 环境,在拥有边界符的同时,可以通过可选参数手动指定列对齐:

\begin{tcolorbox}[sidebyside]
\begin{lstlisting}
\[ \begin{pmatrix*}[r]
    150 & -450 \\ 10 & 15
\end{pmatrix*} \]
\end{lstlisting} 

\tcblower

\[ \begin{pmatrix*}[r]
    150 & -450 \\ 
     10 & 15
\end{pmatrix*} \]
\end{tcolorbox}


\end{enumerate}
\subsection{并列公式}
\begin{enumerate}

\item 分段函数与公式并列:使用 \boxforenv{cases} 环境,会自动生成一个比 \verb|\left| 更紧凑的花括号,例如:

\begin{tcolorbox}[sidebyside]
\begin{lstlisting}
\[ y=\begin{cases}
    x & (x \le 1) \\ 
    2x-1 & (x>1) 
\end{cases}\]
\end{lstlisting} 

\tcblower

$$ y=\begin{cases}x & (x\le1) \\ 2x-1 & (x>1) \end{cases} $$
\end{tcolorbox}

\verb|cases| 环境行列可像矩阵一样对齐,但最多只允许包含两列。

\item 并列公式中的公式样式:\verb|cases| 环境下的公式默认为 text 样式,使用 \boxforpkg{mathtools} 宏包的 \boxforenv{dcases} 环境可以得到 display 样式的内容。以下是两者的比较:

\begin{tcolorbox}[colback=white]
\[ \begin{cases}
    \int_0^x \frac{\dd{}x}{x+1} & x > 0 \\
    x + 1 & x \leq 0
\end{cases} \qquad \begin{dcases}
    \int_0^x \frac{\dd{}x}{x+1} & x > 0 \\
    x + 1 & x \leq 0
\end{dcases} \]
\end{tcolorbox}

\item 左并列与右括号:使用 \boxforpkg{mathtools} 宏包的 \boxforenv{rcases} 环境可以得到右花括号环境,效果为:

\begin{tcolorbox}[sidebyside]
\begin{lstlisting}
\[ \begin{rcases}
    a > 0 \\
    x_1=2,x_2=5
\end{rcases} y=x^2-7x+10 \]
\end{lstlisting} 

\tcblower

\[\quad\begin{rcases}
    a > 0 \\
    x_1=2,x_2=5
\end{rcases} 
y=x^2-7x+10 \]
\end{tcolorbox}
 
\item 并列公式和并列文本:使用带星号的 \boxforenv{dcases*} 环境可以使并列环境的第二列条件不是公式而是文本,例如:

\begin{tcolorbox}[sidebyside]
\begin{lstlisting}
\[a_n =\begin{dcases*} 
    2^n & n is odd\\
    n^2 & n is even
\end{dcases*} \]
\end{lstlisting} 

\tcblower

\[\quad a_n=\begin{dcases*} 
    2^n & n is odd\\
    n^2 & n is even
  \end{dcases*} \]
\end{tcolorbox}


\end{enumerate}
\subsection{公式与编号}
\begin{enumerate}

\item 带编号的公式:普通的行间公式不带编号,使用 \boxforenv{equation} 环境可以创建带编号的公式:

\begin{tcolorbox}[sidebyside]
\begin{lstlisting}
\begin{equation}
    a^2 + b^2 = c^2
\end{equation}
\end{lstlisting} 

\tcblower

\begin{equation}
    a^2 + b^2 = c^2
\end{equation}
\end{tcolorbox}

\verb|equation| 环境自带公式环境,可以直接输入公式。

\item 连续编号的公式:\boxforpkg{amsmath} 宏包的 \boxforenv{gather} 环境可以给多行公式编号,环境内使用两个连续的反斜杠 \boxforcmd{\\\\} 换行,例如:

\begin{tcolorbox}[sidebyside]
\begin{lstlisting}
\begin{gather}
    (a+b)^2=a^2+2ab+b^2\\
    (a-b)^2=a^2-2ab+b^2
\end{gather}
\end{lstlisting} 

\tcblower

\begin{gather}
    (a+b)^2=a^2+2ab+b^2\\
    (a-b)^2=a^2-2ab+b^2
\end{gather}
\end{tcolorbox}

\item 多行的行间公式:普通的行间公式无法这样换行,带星号的 \boxforenv{gather*} 环境可以创建多行的行间公式,且均不编号。

\item 公式与子编号:\boxforpkg{amsmath} 宏包使用 \boxforenv{subequations} 环境可以使环境内的公式编号变为子编号,例如:

\begin{tcolorbox}[sidebyside]
\begin{lstlisting}
\begin{subequations} \begin{gather}
    \sin^2 x=\frac{1-\cos 2x}{2}\\
    \cos^2 x=\frac{1+\cos 2x}{2}
\end{gather} \end{subequations}
\end{lstlisting} 

\tcblower

\begin{subequations} \begin{gather}
    \sin^2 x=\frac{1-\cos 2x}{2}\\
    \cos^2 x=\frac{1+\cos 2x}{2}
    \end{gather} 
\end{subequations}
\end{tcolorbox}

\item 自定义编号行为:在行尾使用命令 \boxforcmd{\\notag} 可以取消该行的编号;在行尾使用命令 \boxforcmd{\\tag{}} 可以手动指定公式的编号,参数内为编号内容,且该编号会自动添加圆括号;带星号的命令 \boxforcmd{\\tag*{}} 同样可以手动指定编号,且不会自动添加圆括号。例如:

\begin{tcolorbox}[sidebyside]
\begin{lstlisting}
\begin{gather}
    a+b                 \\
    b+c \notag          \\
    c+d \tag{ii}        \\
    d+e \tag*{$\star$}  \\
\end{gather}
\end{lstlisting} 

\tcblower

\begin{gather}
    a+b\\
    b+c \notag\\
    c+d \tag{ii}\\
    d+e \tag*{$\star$}
\end{gather}
\end{tcolorbox}


\end{enumerate}

\subsection{对齐公式}

\begin{enumerate}

\item 公式对齐:使用 \boxforpkg{amsmath} 宏包的 \boxforenv{align} 环境,通过 \boxforcmd{&} 符号确定对齐位置:

\begin{tcolorbox}[sidebyside]
\begin{lstlisting}
\begin{align} a(b+c) 
    &= ab + ac \\
    &= ac + ab \\ 
    &= a(c+b) \end{align}
\end{lstlisting} 

\tcblower

\begin{align} a(b+c) &= ab + ac \\
    &= ac + ab \\ &= a(c+b) \end{align}
\end{tcolorbox}

\boxforenv{align} 环境实质是奇数列居右、偶数列居左的表格,因此不用像 \boxforenv{array} 环境需要给出列的数目和对齐参数,并且可以通过空列改变列的左右对齐方式。

\boxforenv{align} 环境会给每一行公式编号,带星的 \boxforenv{align*} 环境会取消所有编号。

\item 对齐与空白:\boxforpkg{amsmath} 宏包的 \boxforenv{align} 环境列间会产生较大的空白,换用 \boxforenv{alignat} 环境可以得到紧凑的对齐。该环境需要指定一个数字作为参数,代表右左对齐的对数,计算方法为 $\geq$ 每行最大的 \verb|&| 数量加 1 后再除以 2 。

使用带星号的 \boxforenv{alignat*} 环境可以不带编号。

例如,以下对齐:

\begin{tcolorbox}[sidebyside]
\begin{lstlisting}
\begin{alignat*}{3}
    x+&2&&y=-&5\\
     -& &&y= &7
\end{alignat*}
\end{lstlisting} 

\tcblower

\begin{alignat*}{3}
    x+&2&&y=-&5\\
     -& &&y= &7
\end{alignat*}
\end{tcolorbox}

换用 \boxforenv{align} 环境会产生大量的空白,效果一言难尽,但可以从中看出 \boxforenv{alignat} 环境的参数数为 \verb|3| :

\begin{tcolorbox}[colback=white]
\vspace{-1em}
\begin{align*}
    x+&2&&y=-&5\\
     -& &&y= &7
\end{align*}
\end{tcolorbox}

\item 公式块环境:\verb|gather|、\verb|align| 和 \verb|alignat| 环境必定占据一整行,公式块环境 \verb|gathered| 、\verb|aligned| 和 \verb|alignated| 环境只占公式实际宽度,因此同一行中可以编写其它内容。但是公式块环境需要置于其它数学环境中,且不带任何编号。

\item 对齐与编号:将公式块环境嵌套在 \verb|equation| 环境中可以使多行公式只有一个居中的编号,例如:

\begin{tcolorbox}[sidebyside]
\begin{lstlisting}
\begin{equation} \begin{aligned}
    e^x&=1+x+\frac{1}{2!}x^2+\cdots
            +\frac{1}{n!}x^n+\cdots\\
    &=\sum_{n=0}^{\infty}\frac{x^n}{n!}
\end{aligned} \end{equation}
\end{lstlisting} 

\tcblower

\begin{equation}
    \begin{aligned}
        e^x&=1+x+\frac{1}{2!}x^2+\cdots+\frac{1}{n!}x^n+\cdots\\
        &=\sum_{n=0}^{\infty}\frac{x^n}{n!}
    \end{aligned}
\end{equation}
\end{tcolorbox}

\item 左中右对齐:\boxforpkg{amsmath} 宏包使用 \boxforenv{multline} 环境得到第一行左对齐、中间的行居中对齐、最后一行右对齐的公式环境:

\begin{tcolorbox}[sidebyside]
\begin{lstlisting}
\begin{multline} 
    a_1 + b_1 \\ 
    a_2 + b_2 \\ 
    a_3 + b_3 \\ 
    a_4 + b_4 
\end{multline}
\end{lstlisting} 

\tcblower

\begin{multline} a_1 + b_1 \\ 
    a_2 + b_2 \\ a_3 + b_3 \\ 
    a_4 + b_4 \end{multline}
\end{tcolorbox}

使用带星号的 \boxforenv{multline*} 环境可以去除最后一行产生的编号。


\end{enumerate}