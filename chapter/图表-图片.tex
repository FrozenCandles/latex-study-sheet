\subsection{图片}

\begin{enumerate}

\item \LaTeX 中的图片:使用插图有两种途径,一是插入外部绘制的图片,二是使用 \LaTeX 代码直接在文档中画图。

\item 插入图片:需要 \boxforpkg{graphicx} 宏包支持,使用 \boxforcmd{\\includegraphics} 命令插图。可选参数调整图片属性,必选参数指明图形的路径:

\begin{tcolorbox}[sidebyside]
\begin{lstlisting}
Hello, \includegraphics
[height=0.5cm]{./LaTeX.png}
\end{lstlisting} 

\tcblower

Hello, \includegraphics
[height=0.5cm]{./resource/graph/LaTeX.png}
\end{tcolorbox}

支持的图形格式包括 PDF、PNG、JPG、EPS 等。

插入的图形就是一个有内容的矩形盒子,在正文中排版效果和一个很大的字符类似。

\item 图片环境:除了一些很小的标志图形,很少把插图直接夹在文字中。通常把图形放在一个可以变动相对位置的浮动环境中,使用 \boxforenv{figure} 环境:

\begin{tcolorbox}[sidebyside]
\begin{lstlisting}
\begin{figure}[ht]
    \centering
    \includegraphics{./LaTeX.png}
    \caption{\LaTeX symbol}
\end{figure}
\end{lstlisting} 

\tcblower

    \centering
    \includegraphics[width=2cm]{./resource/graph/LaTeX.png}\vspace{-10pt}
    \figcaption{\LaTeX symbol}
\end{tcolorbox}

\boxforenv{figure} 环境有可选参数 \boxforcmd{[ht]} ,表示浮动体可以出现在环境周围的文本所在处(here)和一页的顶部(top)。\boxforenv{figure} 环境内部相当于默认没有缩进的段落。

\boxforcmd{\\caption} 命令给插图加上自动编号和标题。%\boxforcmd{\\label} 命令则给图形定义一个标签,用于在文章的其他地方引用该图片。


\end{enumerate}