\section{数学公式}

\subsection{引入公式}

\begin{enumerate}

\item 行内公式:行内公式将嵌入到文本行中,公式垂直距离不会过高。行内公式有 3 种引入方式:

\begin{itemize}
    \item \verb|$...$|
    \item \verb|\(...\)|
    \item \verb|\begin{math}...\end{math}|
\end{itemize}

例如:

\begin{tcolorbox}[sidebyside]
\begin{lstlisting}
$ a + b = c $
\end{lstlisting} 
\tcblower

$ a + b = c $
\end{tcolorbox}

\item 行间公式:行间公式将单独出现并居中,垂直距离将适应公式内容。行间公式有 3 种引入方式:

\begin{itemize}
    \item \verb|$$ ... $$|
    \item \verb|\[ ... \]|
    \item \verb|\begin{displaymath} ... \end{displaymath}|
\end{itemize}

例如:

\begin{tcolorbox}[sidebyside]
\begin{lstlisting}
\[ \sum_{i=1}^{n}x_i \]
\end{lstlisting} 

\tcblower

\[
    \sum_{i=1}^{n}x_i
\]
\end{tcolorbox}

如果用行内公式表达它,受行高限制呈现的效果稍有不同,接下来给出示例:

\item 公式尺寸:公式在不同位置会呈现不同尺寸,如果要让公式强制排版为特定的尺寸,可以使用以下几个声明:

\begin{tcolorbox}[colback=white]
\begin{center}
\begin{tabular}{ccc}
    命令 & 尺寸 & 示例\\
\hline
    \verb| \displaystyle | & 行间公式尺寸 & $\displaystyle{\sum_{i=1}^{n}x_i}$\\
    \verb| \textstyle | & 行内公式尺寸 & $\textstyle{\sum_{i=1}^{n}x_i}$\\
    \verb| \scriptstyle | & 上下标公式尺寸 & $\scriptstyle{\sum_{i=1}^{n}x_i}$\\
    \verb| \scriptscriptstyle | & 次上下标尺寸 & $\scriptscriptstyle{\sum_{i=1}^{n}x_i}$\\
\end{tabular}
\end{center}
\end{tcolorbox}

\item 下标与上标:下标使用 \boxforcmd{_} 符号,上标使用 \boxforcmd{^} 符号,例如 \boxforcmd{a_n} \, $a_n$ 或 \boxforcmd{e^x} \, $e^x$ 

% \begin{center}
% \begin{tabular}{cc}
% \hline
% \quad\verb|a_n| &\quad $a_n$\quad\\
% \quad\verb|e^x| &\quad $e^x$\quad\\
% \hline
% \end{tabular}
% \end{center}

如果上下标内不止一个字符,这些字符需要用花括号 \boxforcmd{\{\}} 定界,例如 \boxforcmd{x^\{10\}} \, $ x^{10} $ 。

\item 数学符号:在数学公式内的字母将会变为斜体。一些基本函数如 \boxforcmd{\\sin} 需要使用命令表示,从而用正体表示字母:

\begin{tcolorbox}[sidebyside]
\begin{lstlisting}
function \sin x
\end{lstlisting} 

\tcblower

$ function \sin x $
\end{tcolorbox}

\end{enumerate}
\subsection{基本公式排版}
\begin{enumerate}

\item 基本符号:加号 \boxforcmd{+} $+$ 减号 \boxforcmd{-} $-$ 和等号 \boxforcmd{=} $=$ 直接使用对应字符创建,其余四则运算符号需要使用命令创建:乘号 \boxforcmd{\\times} $\times$ 除号 \boxforcmd{\\div} $\div$ 不等号 \boxforcmd{\\neq} $\neq$

\item 分式:使用命令 \boxforcmd{\\frac\{\}\{\}} ,两个参数分别表示分子和分母,例如:

\begin{tcolorbox}[sidebyside]
\begin{lstlisting}
\[ \frac{e^x}{\frac a b} \]
\end{lstlisting} 

\tcblower

\[
    \frac{e^x}{\frac a b}
\]
\end{tcolorbox}

同样,如果分子或分母内不止一个字符,需要用花括号定界。

\item 导数:直接使用单引号 \boxforcmd{'} 来表示:

\begin{tcolorbox}[sidebyside]
\begin{lstlisting}
$ f'(x) = a^x \ln x $
\end{lstlisting} 

\tcblower

$ f'(x) = a^x \ln x $
\end{tcolorbox}

\item 三种带上下限的特殊符号:求和 \boxforcmd{\\sum} $\sum$ 、求积 \boxforcmd{\\prod} $\prod$ 、积分 \boxforcmd{\\int} $\int$ ,它们均使用下标 \boxforcmd{_} 符号和上标 \boxforcmd{^} 符号来表示其上下限,例如:

\begin{tcolorbox}[sidebyside]
\begin{lstlisting}
\sum_{i=1}^{N} a_i
\end{lstlisting} 

\tcblower

$$ \sum_{i=1}^{N} a_i $$
\end{tcolorbox}

\item 根号:使用 \boxforcmd{\\sqrt} 命令,配合可选参数可以得到不同次数的根式:

\begin{tcolorbox}[sidebyside]
\begin{lstlisting}
\sqrt{9} \sqrt[3]{x}
\end{lstlisting} 

\tcblower

$$ \sqrt{9} \quad \sqrt[3]{x} $$
\end{tcolorbox}

\item 极限:用 \boxforcmd{\\lim} 表示极限运算符,使用下标的形式表示底下的趋近关系,其中使用 \boxforcmd{\\to} 命令表示趋近的箭头,例如:

\begin{tcolorbox}[sidebyside]
\begin{lstlisting}
\[ \lim_{x \to \infty}f(x) \]
\end{lstlisting} 

\tcblower

\[ \quad \lim_{x \to \infty}f(x) \]
\end{tcolorbox}


\end{enumerate}
\subsection{公式的细节}
\begin{enumerate}

\item 空格:源文件中在数学公式中的空格会被忽略。可以通过以下几种命令向公式中添加空格:

\begin{tcolorbox}[colback=white]
\begin{center}
\begin{tabular}{cccc}
    命令 & 空格大小 & 命令 & 空格大小\\
\hline
    \verb| \, | & $3/18$ 空格 & \verb| \: | & $4/18$ 空格\\
    \verb| \; | & $5/18$ 空格 & \verb| \! | & $-3/18$ 空格\\
    \verb| \quad | & $1$ 空格 & \verb| \qquad | & $2$ 空格\\
    \verb| \|\texttt{\textvisiblespace} & $9/18$ 空格 & (反斜杠加空格)  & \\
\end{tabular}
\end{center} 
\end{tcolorbox}

负数尺寸的空格会拉进两个字符的距离。

\item 数学运算符与数学关系符:数学运算符和数学关系符区别在于两侧的间距。例如 \verb|$a+b$| 表现为 $a+b$ ,但 \verb|$+b$| 表现为 $+b$ 。前者的符号 \verb|+| 是数学运算符,而后者的符号 \verb|+| 是正号符号。数学运算符相比普通符号,左右两侧间距更大。数学关系符类似,且两侧间距比运算符略大,如 \verb|$a<b$| 中的关系符表现为 $a<b$ 。

数学模式中花括号 \verb|{}| 内的公式将独立考虑符号关系,可以与两侧符号隔离,使之不成为运算符或关系符,从而取消间距,如 \verb|$a{+}b$| 表现为 $a{+}b$ 。

普通符号两侧没有预留间距,命令 \boxforcmd{\\mathbin{}} 与 \boxforcmd{\\mathrel{}} 则能分别把参数转换为二元运算符和二元关系符,并正确设置两侧的空距,在对齐时用处较大。

\item 公式与标点符号:公式中常用的标点符号有 \boxforcmd{,;} ,它们与普通符号的区别为只与右侧符号有间距。普通的冒号 \boxforcmd{:} 得到的是数学关系符,而命令 \boxforcmd{\\colon} 得到的才是标点符号中的冒号,且间距左小右大,参见 \boxforcmd{$a:b$} $a:b$ 与 \boxforcmd{$a \colon b$} $a \colon b$ 的区别。

命令 \boxforcmd{\\mathpunct{}} 能把参数作为标点符号处理,并正确设置左无右有的间距。

\item 操作符与上下标:数学操作符的上下标作为行间公式时,将会显示在正下方,而不是右上下方,例如 \boxforcmd{\\min_i} 会显示为 $\displaystyle{\min_i}$ ,\boxforcmd{\\mathop{}} 命令可以将参数转换为操作符,例如 \boxforcmd{x_n^2} $\displaystyle{x_n^2}$ 与 \boxforcmd{\\mathop{x}_n^2} $\displaystyle{\mathop{x}_n^2}$ 的区别

\boxforcmd{\\limits} 命令用在行内公式中,跟随在任意数学操作符后,可以将它的上下标显示在数学操作符的正上下方,并且保持行内公式的紧凑性,例如:

\begin{tcolorbox}[sidebyside]
\begin{lstlisting}
$\int\limits_a^b f(x)\df{}x$
\end{lstlisting} 

\tcblower

${\int\limits_a^b f(x)\mathrm{d}x}$
\end{tcolorbox}

\item 堆叠上下标:\boxforcmd{\\substack{}} 命令需要 \boxforenv{amsmath} 宏包支持,用于将多行符号堆叠为一个上下标,参数中可以使用两个下划线 \boxforcmd{\\\\} 换行,例如:

\begin{tcolorbox}[sidebyside]
\begin{lstlisting}
\[ 
  \lim_{\substack{x \to x_0 \\ 
                  y \to y_0}} f(x, y) 
\]
\end{lstlisting} 

\tcblower

\[ 
    \lim_{\substack{x \to x_0 \\ y \to y_0}} f(x, y) 
\]
\end{tcolorbox}

\item 定界符:以下展示了一些常用的定界符:

\begin{tcolorbox}[colback=white]
\begin{center}
\begin{tabular}{cccc|cccc}
    命令 & 符号 & 命令 & 符号 &
    命令 & 符号 & 命令 & 符号\\
    \verb|(| & $($ & \verb|)| & $)$ \\
    \verb|\lbrace| or \verb|{| & $\lbrace$ & \verb|\rbrace| or \verb|}| & $\rbrace$ &
    \verb|\lbrack| or \verb|[| & $\lbrack$ & \verb|\rbrack| or \verb|]| & $\rbrack$ \\
    \verb|\lfloor| & $\lfloor$ & \verb|\rfloor| & $\rfloor$ &
    \verb|\lceil| & $\lceil$ & \verb|\rceil| & $\rceil$ \\
    \verb|\langle| & $\langle$ & \verb|\rangle| & $\rangle$ &
    \verb|\vert| or \lstinline||| & $\vert$ & \verb|\Vert| or \lstinline|\|| & $\Vert$ \\
\end{tabular}
\end{center}
\end{tcolorbox}

\item 定界符高度:使用 \boxforcmd{\\big} 等一系列命令及其与 \boxforcmd{r} 和 \boxforcmd{l} 的组合可以产生具有不同大小的括号:

\begin{tcolorbox}[sidebyside]
\begin{lstlisting}
$$ ( \big( \Big(
    \frac{ax+b}{\ln\,x}
  \bigg) \Bigg) \Biggr) $$
\end{lstlisting} 

\tcblower

$$ (\big(\Big(
\frac{ax+b}{\ln\,x}
\bigg)\Bigg)\Biggr ) $$
\end{tcolorbox}

使用 \boxforcmd{\\left} 、\boxforcmd{\\right} 以及 \boxforcmd{\\middle} 能使定界符自适应公式的高度。\verb|\left| 和 \verb|\right| 必须成对出现以限定范围。单个点号表示的 \boxforcmd{\\left.} 和 \boxforcmd{\\right.} 仅用于配对以限定范围,不输出任何符号。

大于号 \boxforcmd{<} 和小于号 \boxforcmd{>} 也可以组合这些命令得到不同大小的尖括号。

\item 分式的各种表现:基本的 \verb|\frac| 会在不同位置表现出不同的尺寸,\boxforcmd{\\tfrac} 和 \boxforcmd{\\dfrac} 分别创建 text 和 display 样式的分式,例如 \boxforcmd{\\dfrac{1}{1+x}} 在行内也显示为 $\dfrac{1}{1+x}$ ,这会撑起行高。

\item 连分式:使用 \boxforcmd{\\cfrac} 创建,以下是分别使用 \verb|\frac|、\verb|\dfrac| 和 \verb|\cfrac| 创建的嵌套分式效果:

\begin{tcolorbox}[colback=white]
\[
    a_0+\frac{1}{a_1+\frac{1}{a_2+\frac{1}{a_3}}}
    \qquad
    a_0+\dfrac{1}{a_1+\dfrac{1}{a_2+\dfrac{1}{a_3}}}
    \qquad
    a_0+\cfrac{1}{a_1+\cfrac{1}{a_2+\cfrac{1}{a_3}}}
\]
\end{tcolorbox}


\end{enumerate}
\subsection{公式字体与字符}
\begin{enumerate}

\item 粗体:数学环境中的粗体使用 \boxforpkg{amsmath} 宏包提供的 \boxforcmd{\\boldsymbol\{\}} 命令,例如:

\begin{tcolorbox}[sidebyside]
\begin{lstlisting}
$\boldsymbol{y} = k \boldsymbol{x}$
\end{lstlisting} 

\tcblower

$\boldsymbol{y} = k \boldsymbol{x} $
\end{tcolorbox}

\item 正体:数学公式内的字体默认为斜体,使用命令 \boxforcmd{\\mathrm\{\}} 可以在公式内创建正体,例如 \verb|$\mathrm{argmin}$| 效果为 $\mathrm{argmin}$

\item 原生数学字体:下表列出了原生的数学字体:

\begin{tcolorbox}[colback=white]
\begin{center}
\begin{tabular}{cc}
    命令 & 效果 \\
    \hline
    \verb|\mathrm{ABCDabcd1234}| & $\mathrm{ABCDabcd1234}$ \\
    \verb|\mathit{ABCDabcd1234}| & $\mathit{ABCDabcd1234}$ \\
    \verb|\mathnormal{ABCDabcd1234}| & $\mathnormal{ABCDabcd1234}$ \\
    \verb|\mathcal{ABCDabcd1234}| & $\mathcal{ABCDabcd1234}$ \\
\end{tabular}
\end{center}
\end{tcolorbox}

\item 其它数学字体:其它一些宏包支持的数学字体对 \verb|ABCDabcd1234| 的排版效果为:

\begin{tcolorbox}[colback=white]
\begin{center}
\begin{tabular}{ccc}
    命令 & 效果 & 所需宏包 \\
    \hline
    \verb|\mathbb| & $\mathbb{ABCDabcd1234}$ & \verb|amsfonts|/\verb|amssymb| \\
    \verb|\mathfrak| & $\mathfrak{ABCDabcd1234}$ & \verb|amsfonts|/\verb|amssymb| \\
    \verb|\mathscr| & $\mathscr{ABCDabcd1234}$ & \verb|mathrsfs| \\
\end{tabular}
\end{center}
\end{tcolorbox}

\item 希腊字母:使用对应字母名表示,例如 \boxforcmd{\\alpha} $\alpha$ ,首字母大写表示大写希腊字母,例如 \boxforcmd{\\Omega} $\Omega$ 。有些希腊字母前面加上 var 表示花写,例如 \boxforcmd{\\varphi} $\varphi$ 相比于 \boxforcmd{\\phi} $\phi$ 。

\item 正体小写希腊字母:宏包 \boxforpkg{txfonts} 以 \verb|up| 结尾的命令可以得到正体的小写希腊字母,如 \boxforcmd{\\thetaup} $\uptheta$ 相较于斜体的 \boxforcmd{\\theta} $\theta$ 。

宏包 \boxforpkg{txfonts} 可能会改变其它数学符号甚至正文的字体样式,可以替换为宏包 \verb|upgreek| ,它以 \verb|up| 开头的命令也可以得到正体小写希腊字母,如 \boxforcmd{\\uppi} $\uppi$ ,且不会有副作用。宏包 \verb|upgreek| 有三个互斥的可选项:\verb|Euler|、\verb|Symbol| 和 \verb|Symbolsmallscale| ,分别加载不同的字体或字号。


\end{enumerate}