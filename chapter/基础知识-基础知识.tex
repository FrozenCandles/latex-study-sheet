\subsection{\LaTeX 文档结构}
\begin{enumerate}

\item 文档结构:以下是一份最为简单的 \LaTeX 文档:

\begin{tcolorbox}[sidebyside]
\begin{lstlisting}
\documentclass{article} 
\begin{document}
Hello, \LaTeX
\end{document}
\end{lstlisting} 

\tcblower

\begin{center}
Hello, \LaTeX
\end{center}    
\end{tcolorbox}

\item 命令:在 \LaTeX 中,命令也称控制序列(control sequence),以一个反斜杠加上命令名构成。开头的命令 \newline \boxforcmd{\\documentclass} 指定了使用的文档类,花括号 \boxforcmd{\{\}} 内的内容是命令的参数,\verb|article| 表示文章格式的文档类,除此之外可用的参数还有表示报告的文档类 \verb|report| 和表示书籍的文档类 \verb|book| 等。

如果命令的参数只有一个字符,则花括号可省略,但需要用空格区分命令与参数。

有些命令还存在可选参数,可选参数通常会在花括号前用方括号 \boxforcmd{[]} 表示,例如:

\begin{tcolorbox}
\begin{lstlisting}
\documentclass[11pt,a4paper]{article} 
\end{lstlisting} 
\end{tcolorbox}

多数命令只在原地产生效果,但有些命令则会影响作用域内后面的所有内容,这种命令又称为声明(declaration)。

\boxforcmd{\\begin} 和 \boxforcmd{\\end} 一对命令定义了一个环境,环境表示命令或内容的作用范围。环境如果有备选或额外参数,只需在 \verb|\begin| 中表示。\boxforenv{document} 环境当中的内容是文档正文,在此环境外书写的内容可能不会出现在文档中。

语句 \verb|\begin{document}| 之前的内容称为\textbf{导言区},导言区可以留空,也可以编写文档所需要的信息与工具。

\verb|\LaTeX| 命令用于输出一个特殊的符号 \LaTeX ,像这样表示符号的命令还有很多。

\item 注释:以百分号 \boxforcmdbox{\lstinline[language=Python]|\%|} 开头的部分是行注释。

注释有时放在行尾,用于取消换行产生的一个多余的空格。

如果要单独表示百分号,需要使用反斜杠转义 \boxforcmdbox{\vphantom{f}\textbackslash\lstinline[language=Python]|\%|}

\item 单位:\LaTeX 中常用的衡量长度的单位有:

\begin{itemize}
    \item pt :磅(point)
    \item pc :四号字(pica) 1 pc = 12 pt
    \item in :英寸(inch) 1 in = 72.27 pt 
    \item bp :大点(bigpoint) 1 bp = $\displaystyle{\frac{1}{72}}$ in 
    \item cm :厘米(centimeter) 1 cm = $\displaystyle{\frac{1}{2.54}}$ in
    \item mm :毫米(millimeter)
    \item sp :\TeX 的基本长度单位 scaled point ,1 sp = $\displaystyle{\frac{1}{65536}}$ pt
    \item em :当前字号下大写字母 M 的宽度
    \item ex :当前字号下小写字母 x 的高度
\end{itemize}

\item 宏包:宏包的作用是扩展或调整 \LaTeX 的排版功能。一个宏包往往能提供更多的命令、环境,或为内置的命令/环境添加更多功能。

在导言区使用 \boxforcmd{\\usepackage{}} 即可引入宏包,然后便可以使用宏包提供的功能。部分宏包在引入时可以通过命令的可选参数调节需要引入的功能。

例如,引入 \boxforpkg{ctex} 宏包可以在文档中排版中文字符,引入 \boxforpkg{amssymb} 宏包可以使用命令 \verb|\bigstar| 表示一个填充五角星符号 $\bigstar$



\end{enumerate}
