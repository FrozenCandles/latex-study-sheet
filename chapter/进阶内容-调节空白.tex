\subsection{调节空白距离}

\begin{enumerate}

\item 定长水平空白:以下几个水平空白在公式环境外也能产生相同的空白效果:

\begin{tcolorbox}[colback=white]
\begin{lstlisting}
\, \: \; \! \quad \qquad
\end{lstlisting}
\end{tcolorbox}

这些命令置于段首和段尾无效。

\item 自定义水平空白:\boxforcmd{\\hspace{}} 用于在两个字符间产生高度为零的水平空白,参数为产生空白的水平距离。水平距离可以为负值,表示拉近间距。

该命令置于段首有效,置于段尾无效。如果命令位于换行处(左右字符输出后位于行尾和行首)将失效。

带星号的 \boxforcmd{\\hspace*{}} 作用基本相同,但它在换行处也可以工作,一定能在其左右字符之间生成空白。

\item 虚位水平空白:\boxforcmd{\\hphantom{}} 同样在两个字符之间产生空白,但空白宽度为参数中内容的水平宽度。换句话说它为参数的内容“预留水平空白”。该命令置于段首和段尾无效。

\item 弹性水平空白:\boxforcmd{\\hfill} 命令用于生成弹性水平空白,用于将当前行剩余的空间填满。如果当前行已经充满内容则无效。

\verb|\hfill| 的这些衍生命令可以使用具体的形状而不是空白填充:

\begin{tcolorbox}[colback=white]
\begin{center}
\begin{tabular}{ll}
    \mbox{} \hfill 命令 \hfill \mbox{} & \mbox{} \hfill 效果 \hfill \mbox{} \\
\hline
    \verb|\dotfill| &  用点线填充 \\
    \verb|\hrulefill| &  用水平线段填充 \\
    \verb|\downbracefill| &  用开口向下的花括号填充 \\
    \verb|\upbracefill| &  用开口向上的花括号填充 \\
    \verb|\leftarrowfill| &  用向左的箭头填充 \\
    \verb|\rightarrowfill| &  用向右的箭头填充 \\
\end{tabular}
\end{center}
\end{tcolorbox}

效果示例:

\begin{tcolorbox}[sidebyside]
\begin{lstlisting}
chapter1.3 \dotfill filling \\
left \leftarrowfill mid 
     \rightarrowfill right
\end{lstlisting} 

\tcblower

chapter1.3 \dotfill filling\\
left \leftarrowfill mid \rightarrowfill right
\end{tcolorbox}

\item 自定义竖直空白:\boxforcmd{\\vspace{}} 用于在两段间产生宽度为零的竖直空白,参数为产生空白的竖直距离,负值表示拉近间距。

该命令置于页首或页尾无效。

带星号的 \boxforcmd{\\hspace*{}} 作用基本相同,但置于页首或页尾仍有效,一定能在其上下内容之间生成空白。

\item 定长竖直空白:\boxforcmd{\\smallskip} 生成一段高度为 $3^{+1}_{-1}$ 的可伸缩的垂直空白,输出时系统会自动选择范围内的一个值。类似地,\boxforcmd{\\medskip} 生成一段高度为 $6^{+2}_{-2}$ 的可伸缩的垂直空白;\boxforcmd{\\bigskip} 生成一段高度为 $12^{+4}_{-4}$ 的可伸缩的垂直空白。

\item 虚位竖直空白:\boxforcmd{\\vphantom{}} 在两段之间产生空白,但空白高度为参数中内容的竖直高度。它为参数的内容“预留竖直空白”。

该命令用在数学环境中,可主动调节 \verb|\left \right| 的自适应大小。

\item 弹性竖直空白:\boxforcmd{\\vfill} 命令将当前页面剩余的垂直空间填满。该命令置于页首无效。

\end{enumerate}